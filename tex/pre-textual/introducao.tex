\section{Contextualização}

O dióxido de carbono (CO\textsubscript{2}) é um gás incolor e inodoro presente naturalmente na atmosfera e essencial para a vida na Terra. Na agricultura, o CO\textsubscript{2} é fundamental para o processo de fotossíntese, onde as plantas o utilizam para produzir açúcares e oxigênio \cite{ipcc2019land}.

Contudo, o aumento das concentrações de CO\textsubscript{2} causado pelas atividades humanas contribui significativamente para o aquecimento global, intensificando as mudanças climáticas, a alteração nos eventos climáticos e os padrões de precipitação. Esse aumento da temperatura no planeta resulta na acidificação dos oceanos, na perda de biodiversidade e na desertificação, além de impactar negativamente a agricultura e a segurança alimentar \cite{ipcc2021ar6}. Outro ponto importante são as práticas agrícolas, como o desmatamento e o uso intensivo de fertilizantes, que intensificam as emissões de outros gases de efeito estufa, como o metano (CH\textsubscript{4}) e o óxido nitroso (N\textsubscript{2}O). Reduzir as emissões de CO\textsubscript{2} na agricultura é, portanto, essencial para garantir a sustentabilidade do planeta \cite{ipcc2014afolu}.

Nesse contexto, o mercado de carbono emerge como uma ferramenta para incentivar a redução das emissões de gases de efeito estufa. Ao atribuir um valor econômico ao CO\textsubscript{2} e outros gases, o mercado de carbono promove práticas agrícolas mais sustentáveis, permitindo que os produtores rurais participem ativamente na mitigação das mudanças climáticas \cite{worldbank2022carbon}. Esse mecanismo não só oferece incentivos financeiros para reduzir as emissões, mas também contribui para a conservação dos recursos naturais e a proteção dos ecossistemas. Integrar a agricultura a este mercado pode ser uma estratégia eficaz para alinhar a produção agrícola com as metas globais de redução de emissões, destacando a importância de práticas sustentáveis no setor \cite{unfccc2020markets}.

O monitoramento das emissões de dióxido de carbono é crucial para entender e mitigar os impactos das mudanças climáticas. A coleta contínua e a análise de dados permitem identificar as principais fontes de emissão e suas variações, facilitando o desenvolvimento de estratégias eficazes e a promoção de práticas agrícolas sustentáveis \cite{ipcc2014afolu}. Além disso, esses dados são fundamentais para a formulação de políticas públicas voltadas para a redução dos gases de efeito estufa.

Diversas tecnologias são empregadas no monitoramento ambiental, com potencial significativo de aplicação no contexto agrícola. Sensores projetados para medir a qualidade do ar detectam os gases de efeito estufa presentes no ambiente, fornecendo dados importantes para o monitoramento e análise das emissões. Redes mesh de sensores conectam múltiplos dispositivos de forma colaborativa, possibilitando a coleta e transmissão de dados de maneira eficiente e abrangente \cite{zhao2018wsn}. Tecnologias de comunicação sem fio, como o protocolo ESP-MESH utilizado com o módulo ESP8266, permitem a formação de redes mesh que facilitam a comunicação eficiente entre sensores e a transmissão de dados para análise e armazenamento \cite{espressif2024mesh}. Essa abordagem é especialmente útil em aplicações que exigem baixo consumo de energia e cobertura em áreas extensas, como no monitoramento ambiental de gases de efeito estufa. Além disso, essa tecnologia pode desempenhar um papel crucial na regulamentação do mercado de carbono no Brasil, que atualmente é realizada de forma voluntária.

Entretanto, a medição das emissões enfrenta desafios como a precisão dos sensores, que pode ser influenciada por condições ambientais variáveis, e a necessidade de cobertura e conectividade em áreas agrícolas extensas. Além disso, o armazenamento e a análise de grandes volumes de dados exigem sistemas robustos e técnicas avançadas para extrair informações úteis \cite{liu2018iot}.

Este estudo propõe desenvolver e implementar um nó sensorial para o monitoramento das emissões de CO\textsubscript{2} na agricultura. A proposta engloba a criação de um protótipo que integra sensores ambientais a uma rede de comunicação sem fio baseada no protocolo ESP-MESH do microcontrolador ESP8266, permitindo coleta e transmissão de dados. A rede será composta por nós distribuídos estrategicamente em áreas agrícolas, garantindo cobertura abrangente graças à possibilidade de comunicação sem fio e à capacidade do protocolo ESP-MESH de estabelecer conexão entre múltiplos dispositivos, sem depender de uma infraestrutura centralizada \cite{espressif2024mesh}. Essa característica é especialmente vantajosa em ambientes agrícolas, permitindo a instalação de nós sensoriais em lavouras, pastagens e florestas, mesmo em terrenos irregulares, formando uma malha auto-organizável adaptada à topografia local. A distribuição estratégica dos sensores possibilita o monitoramento de grandes extensões. Além disso, a solução apresenta potencial de escalabilidade para outros contextos, como áreas urbanas, priorizando baixo consumo de energia e robustez, alinhando-se às demandas por sustentabilidade e eficiência em diferentes setores.

\section{Objetivo principal}
O objetivo principal deste trabalho é desenvolver e implementar uma rede de nós sensoriais para monitoramento contínuo e autônomo das emissões de CO\textsubscript{2} em cultivos agrícolas.

\section{Objetivos específicos}
\begin{itemize}
    \item Projetar e construir unidades sensoriais autônomas para medição de CO\textsubscript{2}
    \item Implementar a rede de sensores em uma cultura agrícola específica
    \item Coletar e transmitir os dados ambientais para uma plataforma digital
    \item Desenvolver visualizações gráficas intuitivas dos dados coletados
\end{itemize}

\section{Justificativa}

Este estudo é justificado pela relevância do monitoramento das emissões de CO\textsubscript{2} na agricultura como instrumento para incentivar a mitigação das mudanças climáticas e para apoiar a criação de um mercado de carbono regulamentado no Brasil, uma vez que a medição fornecida pelos sensores trará maior transparência ao processo, permitindo a quantificação dessas emissões. A pesquisa concentra-se no desenvolvimento e implementação de tecnologias de sensores e comunicação para a obtenção de dados contínuos sobre as emissões de CO\textsubscript{2}.

O monitoramento das emissões é fundamental para compreender o impacto das práticas agrícolas no meio ambiente e para subsidiar o desenvolvimento de estratégias de mitigação adequadas. Este estudo busca oferecer uma solução prática e eficiente para a coleta e análise de dados de emissões, contribuindo para práticas agrícolas mais sustentáveis e para a formulação de políticas públicas voltadas ao controle de gases de efeito estufa.